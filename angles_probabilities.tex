\documentclass[12pt]{article} % размер шрифта

\usepackage{tikz} % картинки в tikz
\usepackage{microtype} % свешивание пунктуации

\usepackage{array} % для столбцов фиксированной ширины

\usepackage{url} % для вставки ссылок \url{...}

\usepackage{indentfirst} % отступ в первом параграфе

\usepackage{sectsty} % для центрирования названий частей
\allsectionsfont{\centering} % приказываем центрировать все sections

\usepackage{amsthm} % теоремы и доказательства

\theoremstyle{definition} % прямой шрифт в условии теорем
\newtheorem{theorem}{Теорема}[section]


\usepackage{amsmath, amssymb} % куча стандартных плюшек
\usepackage{physics}

\usepackage[top=2cm, left=1.5cm, right=1.5cm, bottom=2cm]{geometry} % размер текста на странице

\usepackage{lastpage} % чтобы узнать номер последней страницы

\usepackage{enumitem} % дополнительные плюшки для списков
%  например \begin{enumerate}[resume] позволяет продолжить нумерацию в новом списке
\usepackage{caption} % подписи к картинкам без плавающего окружения figure


\usepackage{fancyhdr} % весёлые колонтитулы
\pagestyle{fancy}
\lhead{Углы, вероятности и случайные проекции}
\chead{}
\rhead{2021-07-02}
\lfoot{}
\cfoot{}
\rfoot{}
\renewcommand{\headrulewidth}{0.4pt}
\renewcommand{\footrulewidth}{0.4pt}



\usepackage{todonotes} % для вставки в документ заметок о том, что осталось сделать
% \todo{Здесь надо коэффициенты исправить}
% \missingfigure{Здесь будет картина Последний день Помпеи}
% команда \listoftodos — печатает все поставленные \todo'шки

\usepackage{booktabs} % красивые таблицы
% заповеди из документации:
% 1. Не используйте вертикальные линии
% 2. Не используйте двойные линии
% 3. Единицы измерения помещайте в шапку таблицы
% 4. Не сокращайте .1 вместо 0.1
% 5. Повторяющееся значение повторяйте, а не говорите "то же"

\usepackage{fontspec} % поддержка разных шрифтов
\usepackage{polyglossia} % поддержка разных языков

\setmainlanguage{russian}
\setotherlanguages{english}

\setmainfont{Linux Libertine O} % выбираем шрифт
% если Linux Libertine не установлен, то
% можно также попробовать Helvetica, Arial, Cambria и т.Д.

% чтобы использовать шрифт Linux Libertine на личном компе,
% его надо предварительно скачать по ссылке
% http://www.linuxlibertine.org/index.php?id=91&L=1

% на сервисах типа sharelatex.com этот шрифт есть :)

\newfontfamily{\cyrillicfonttt}{Linux Libertine O}
% пояснение зачем нужно шаманство с \newfontfamily
% http://tex.stackexchange.com/questions/91507/



\usepackage[bibencoding = auto,
backend = biber,
sorting = none,
style=alphabetic]{biblatex}

\addbibresource{probability_pro.bib}





\AddEnumerateCounter{\asbuk}{\russian@alph}{щ} % для списков с русскими буквами
\setenumerate[2]{label=\asbuk*)}

%% эконометрические и вероятностные сокращения
\DeclareMathOperator{\Cov}{Cov}
\DeclareMathOperator{\Corr}{Corr}
\DeclareMathOperator{\Var}{Var}
\DeclareMathOperator{\E}{E}
\DeclareMathOperator{\plim}{plim}
\DeclareMathOperator{\card}{card}


\newcommand \hb{\hat{\beta}}
\newcommand \hs{\hat{\sigma}}
\newcommand \htheta{\hat{\theta}}
\newcommand \s{\sigma}
\newcommand \hy{\hat{y}}
\newcommand \hY{\hat{Y}}
\newcommand \vunit{\vec{1}}
\newcommand \e{\varepsilon}
\newcommand \he{\hat{\e}}
\newcommand \z{z}
\newcommand \hVar{\widehat{\Var}}
\newcommand \hCorr{\widehat{\Corr}}
\newcommand \hCov{\widehat{\Cov}}
\newcommand \cN{\mathcal{N}}
\let\P\relax
\DeclareMathOperator{\P}{P}




\begin{document}


Из этого рассказа любопытный читатель узнает, как с помощью проекции на случайную прямую или плоскость ответить на три вопроса:

\begin{enumerate}
\item Чему равна сумма углов в плоском треугольнике?

\item Чему равна сумма углов в сферическом треугольнике?

\item Как связаны двугранные и трехгранные углы в произвольном выпуклом многограннике?
\end{enumerate}


\section{Новое доказательство старой формулы}

Начнём с необычного вероятностного доказательства формулы для суммы углов треугольника. 

Рассмотрим закрашенный треугольник $\Delta ABC$ на плоскости. 

% картинка

Для любой точки плоскости $x$ определим величину $\alpha(x)$. 
Величина $\alpha(x)$ показывает, какой процент территории неподалёку от точки $x$ попадает в треугольник. 

Формально понятие «рядом» задается термином $\varepsilon$-окрестность. 
Вспомним, что $\varepsilon$-окрестностью произвольной точки $x$ на плоскости называют круг радиуса $\varepsilon$ с центром в точке $x$.

И величина $\alpha(x)$ равна доле площади пересечения маленькой $\varepsilon$-окрестности 
с треугольником от общей площади окрестности. 

% картинка


Точка $a$ лежит строго внутри треугольника $\Delta ABC$, 
и для неё доля $\alpha(a)$ равна $1$.

Точка $b$ лежит на стороне треугольника, но не совпадает ни с одной вершиной, 
и для неё доля $\alpha(b)$ равна $0.5$.

Точка $c$ является вершиной треугольника, поэтому для неё доля $\alpha(c)$
равна отношению угла треугольника $\gamma_c$ к $2\pi$.

Точка $d$ лежит вне треугольника, поэтому для неёё доля $\alpha(d)$ равна $0$.


Спроецируем треугольник на случайную прямую. 
Для определённости поясним, как мы выбираем случайную прямую. 
Будем считать, что прямая проходит через начало координат, а направление прямой выбирается равномерно из всех возможных.

В результате закрашенный треугольник с тремя вершинами спроецируется в отрезок с двумя вершинами.
Каждая вершина треугольника либо «выживает» при проецировании и оказывается вершиной отрезка,
либо «погибает», попадая внутрь отрезка.

% картинка

Введем случайную величину $I_v$ — индикатор выживания вершины $v$ при проецировании.
Случайная величина $I_v$ равна 1, если вершина $v$ выживает после проецирования и равна 0, если 
вершина $v$ погибает при процировании. 

Давайте найдём вероятность выживания для вершины треугольника.
На картинке изображены две ситуации: слева вершина выживает при проецировании, справа — погибает.

% картинка

Вершина погибает, если и только если вектор-перпендикуляр к случайной прямой попадает в закрашенную область.
Таким образом, $\P(I_v = 0) = 2 \alpha(v)$ и $\P(I_v = 1) = 1 - 2\alpha(v)$.

А теперь снова посмотрим на суммарное число выживших после проецирования вершин.
Обозначим его $N_V$. 
При проецировании треугольник превращается в отрезок ровно в двумя вершинами, поэтому

\[
N_V = 2
\]

Количество $N_V$ можно разложить в сумму индикаторов выживания каждой из вершин,

\[
\sum_{v \in V} I_v = 2
\]
Здесь $V$ — множество всех вершин треугольника, $v$ — конкретная вершина.


Возьмем математическое ожидание правой и левой частей этого тождества
\[
\E(\sum_{v \in V} I_v) = 2
\]

В силу линейности математического ожидания
\[
\sum_{v\in V} \E(I_v) = 2
\]

Индикаторы принимают только значения $0$ или $1$, поэтому $\E(I_v) = \P(I_v = 1) = 1 - 2\alpha(v)$.

\[
\sum_{v\in V} (1 - 2\alpha(v)) = 2
\]

Или 
\[
\card V = 2 + 2 \sum_{v\in V} \alpha(v), 
\]
где $\card V$ — общее число вершин.

У треугольника ровно 3 вершины, поэтому равенство упрощается до
\[
1 = 2\alpha(v_1) + 2\alpha(v_2) + 2\alpha(v_3)
\]

Поскольку для вершины $\alpha(v) = \gamma / 2\pi$, получаем сумму всех углов треугольника:
\[
\pi = \gamma_1 + \gamma_2 + \gamma_3
\]

Ура, сумма углов треугольнике равна $180^{\circ}$!

Заметим, что все наши рассуждения до формулы
\[
\card V = 2 + 2 \sum_{v\in V} \alpha(v), 
\]
остаются справедливыми и в случае произвольного выпуклого $n$-угольника. 


\section{Сферический треугольник}

Испольуем случайную проекцию и разложение в сумму для сферического треугольника!

Нарисуем сферический треугольник и рассмотрим бесконечный закрашенный внутри трехгранный угол, 
образованный центром сферы и вершинами сферического треугольника. 

% картинка

Для любой точки пространства $x$ определим величину $\alpha(x)$. 
Величина $\alpha(x)$ показывает, какой процент пространства неподалёку от точки $x$ попадает в закрашенный трехранный угол.

Формально $\alpha(x)$ равна доле объёма пересечения маленькой $\varepsilon$-окрестности 
с трёхгранным уголом от общего объёма окрестности. 

Рассмотрим примеры $\alpha(x)$:

% картинка

Точка $a$ лежит строго внутри трёхгранного угла, 
и для неё доля $\alpha(a)$ равна $1$.

Точка $b$ лежит на грани трёхгранного угла, 
но не лежит ни на одном ребре, 
и для неё доля $\alpha(b)$ равна $0.5$.

Точка $d$ лежит вне трёхгранного угла, 
поэтому для неёё доля $\alpha(d)$ равна $0$.

Если точка $c$ лежит на ребре, то для неё доля $\alpha(c)$
равна отношению двугранного угла $\gamma_c$ к $2\pi$.

% картинка

И последний интересный случай, вершина трёхгранного угла, точка $o$:

% картинка 

В силу подобия маленькой $\varepsilon$-сферы и единичной сферы
\[
\alpha(o) = \frac{S}{4\pi},
\]
где $S$ — площадь сферического треугольника. 

Спроецируем бесконечный трёхгранный угол на случайную плоскости. 
Случайную плоскость будем выбирать так же, как и случайную прямую на плоскости. 
Проложим плоскость через начало координат, а направление плоскости выбираем равномерно из всех возможных. 
Можно равномерно выбрать случайную точку $s$ на сфере и провести 
через центр сферы плоскость, перпендикулярную вектору $os$.

В результате проецирования трёхгранный угол превратится 
либо в бесконечный угол на плоскости, либо накроет всю плоскость. 

% картинка

Вершина трёхгранного угла либо выживает при проецировании и оказывается вершиной плоского угла, либо погибает. 

Каждое ребро трёхгранного угла либо выживает при проецировании и оказывается ребром плоского угла, либо погибает, оказываясь внутри плоского угла. 



Вершина погибает, если и только если вектор-перпендикуляр к случайной плоскости попадает в закрашенную область.
Таким образом, $\P(I_v = 0) = 2 \alpha(v)$ и $\P(I_v = 1) = 1 - 2\alpha(v)$.

% картинка

Ребро погибает, если и только если вектор-перпендикуляр к случайной плоскости попадает в закрашенную область.
Таким образом, $\P(I_e = 0) = 2 \alpha(e)$ и $\P(I_e = 1) = 1 - 2\alpha(e)$.

% картинка



Посмотрим на возможное количество выживших рёбер и вершин. 
Если проекция накрывает всю плоскость, то $N_V = 0$ и $N_E = 0$.
Если при проецировании получается плоский угол, то $N_V = 1$ и $N_E = 2$.

В любом случае выполнено соотношение

\[
N_E = 2 N_V
\]


Снова раскладываем количества выживших в сумму:
\[
\sum_{e\in E} I_e = 2\sum_{v\in V} I_v
\]

Снова берём математическо ожидание:
\[
\sum_{e\in E} \E(I_e) = 2\sum_{v\in V} \E(I_v)
\]

Снова переходим к вероятностям выживания:
\[
\sum_{e\in E} 1-2\E(I_e) = 2\sum_{v\in V} \E(I_v)
\]

И получаем новое тождество:
\[
\card V + 2\sum_{e\in E} \alpha(e) = \card E + 2\sum_{v\in V} \alpha(v)
\]





\section{Выпуклый многогранник}


И теперь ничто не остановит нас от произвольного выпуклого многогранника!


Спроецируем произвольный выпуклый многогранник на случайную плоскость.
Получим выпуклый многоугольник. 

% картинка

Про многоугольник мы можем лишь утверждать, 
что у него число вершин равно числу ребёр:

\[
N_V = N_E
\]

Снова раскладываем количество выживших вершин или рёбер в сумму индикаторов:
\[
\sum_{v\in V} I_v = \sum_{e\in E} I_e
\]

Снова берём математическо ожидание:
\[
\sum_{v\in V} \E(I_v) = \sum_{e\in E} \E(I_e)
\]

Снова переходим к вероятностям выживания:
\[
\sum_{v\in V} 1 - 2\alpha(v) = \sum_{e\in E} 1 - 2\alpha(e)
\]

И получаем новое тождество:
\[
\card V + 2\sum_{e\in E} \alpha(e) = \card E + 2\sum_{v\in V} \alpha(v)
\]


\section{Обзор результатов}

Подведём итоги трёх задачек! 

Все они основаны на проецировании объекта на плоскость или прямую и обнаружении некоторого тождества для числа вершин и рёбер у проекции.
Математическое ожидание, взятое от тождества, позволяет перейти к вероятностям выживания вершин или ребёр при проецировании,
а вероятности выживания могут быть проинтерпретированы с помощью углов.


Обозначения общие: 

\begin{itemize}
  \item $V$ — множество вершин исходного многоугольника или многогранника, 
  \item $E$ — множество ребёр, 
  \item $N_V$ — число вершин, выживших при проецировании,
  \item $N_E$ — число рёбер, выживших при проецировании,
  \item $\alpha(v)$ или $\alpha(e)$ — вероятность выживания вершины или ребра.
\end{itemize}


Полученные результаты:

\begin{itemize}
\item  Выпуклый многоугольник:

Тождество для проекции: $N_V = 2$.

Соотношение для углов многоугольника: $\card V = 2 + 2\sum_v \alpha(v)$.

\item Выпуклый многоугольник на сфере:

Тождество для проекции: $N_E = 2 N_V$.

Соотношение для углов многоугольника: $\card E $.

\item Выпуклый многогранник:

Тождество для проекции: $N_V = N_E$.

Соотношение для углов многогранника: $\card V + 2\sum_e \alpha(e) = \card E + 2\sum_v \alpha(v)$.
\end{itemize}



Научной новизны эта статья не содержит и полностью основана на статьях \autocite{klain2020probabilistic} и \autocite{welzl1994gram} из American Mathematical Monthly.
Мне лишь показалось, что для ясности можно изложить ту же идею сначала для обычного треугольника
и добавить поясняющих картинок.



% \section{Источники мудрости}


\printbibliography[heading=none]




\end{document}